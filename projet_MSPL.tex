% Options for packages loaded elsewhere
\PassOptionsToPackage{unicode}{hyperref}
\PassOptionsToPackage{hyphens}{url}
%
\documentclass[
]{article}
\usepackage{amsmath,amssymb}
\usepackage{iftex}
\ifPDFTeX
  \usepackage[T1]{fontenc}
  \usepackage[utf8]{inputenc}
  \usepackage{textcomp} % provide euro and other symbols
\else % if luatex or xetex
  \usepackage{unicode-math} % this also loads fontspec
  \defaultfontfeatures{Scale=MatchLowercase}
  \defaultfontfeatures[\rmfamily]{Ligatures=TeX,Scale=1}
\fi
\usepackage{lmodern}
\ifPDFTeX\else
  % xetex/luatex font selection
\fi
% Use upquote if available, for straight quotes in verbatim environments
\IfFileExists{upquote.sty}{\usepackage{upquote}}{}
\IfFileExists{microtype.sty}{% use microtype if available
  \usepackage[]{microtype}
  \UseMicrotypeSet[protrusion]{basicmath} % disable protrusion for tt fonts
}{}
\makeatletter
\@ifundefined{KOMAClassName}{% if non-KOMA class
  \IfFileExists{parskip.sty}{%
    \usepackage{parskip}
  }{% else
    \setlength{\parindent}{0pt}
    \setlength{\parskip}{6pt plus 2pt minus 1pt}}
}{% if KOMA class
  \KOMAoptions{parskip=half}}
\makeatother
\usepackage{xcolor}
\usepackage[margin=1in]{geometry}
\usepackage{graphicx}
\makeatletter
\def\maxwidth{\ifdim\Gin@nat@width>\linewidth\linewidth\else\Gin@nat@width\fi}
\def\maxheight{\ifdim\Gin@nat@height>\textheight\textheight\else\Gin@nat@height\fi}
\makeatother
% Scale images if necessary, so that they will not overflow the page
% margins by default, and it is still possible to overwrite the defaults
% using explicit options in \includegraphics[width, height, ...]{}
\setkeys{Gin}{width=\maxwidth,height=\maxheight,keepaspectratio}
% Set default figure placement to htbp
\makeatletter
\def\fps@figure{htbp}
\makeatother
\setlength{\emergencystretch}{3em} % prevent overfull lines
\providecommand{\tightlist}{%
  \setlength{\itemsep}{0pt}\setlength{\parskip}{0pt}}
\setcounter{secnumdepth}{-\maxdimen} % remove section numbering
\ifLuaTeX
  \usepackage{selnolig}  % disable illegal ligatures
\fi
\usepackage{bookmark}
\IfFileExists{xurl.sty}{\usepackage{xurl}}{} % add URL line breaks if available
\urlstyle{same}
\hypersetup{
  pdftitle={Factors Influencing Olympic Performance Across Countries},
  hidelinks,
  pdfcreator={LaTeX via pandoc}}

\title{Factors Influencing Olympic Performance Across Countries}
\author{}
\date{\vspace{-2.5em}2025-04-14}

\begin{document}
\maketitle

{
\setcounter{tocdepth}{2}
\tableofcontents
}
\textbf{Students: Rami BELGUITH , Mahrez Mustapha , Cheikh Anis , Saidou
Momo Abdoul kader }

\subsection{1. Introduction}\label{introduction}

This project explores the factors that influence a country's performance
in the Olympic Games. By analyzing historical Olympic athlete data
alongside country-level information, we aim to understand what drives
national success in terms of medal counts. While particular attention is
given to economic indicators such as GDP per capita and population, we
also consider broader historical and geopolitical influences, including
events like World Wars, the Cold War, and national investment in sports.
This approach helps explain why some countries consistently perform
better than others on the global Olympic stage.

\subsubsection{Context \& Dataset
Description}\label{context-dataset-description}

\begin{itemize}
\tightlist
\item
  \textbf{Olympic Dataset}: \texttt{athlete\_events.csv} from Kaggle
  contains data on all athletes who participated in the Olympic Games
  from 1896 to 2016.
\item
  \textbf{GDP \& Population}: Extracted from the World Bank,
  representing GDP per capita and population for most countries between
  1960 and 2020.
\end{itemize}

\subsubsection{How the Dataset Has Been
Obtained?}\label{how-the-dataset-has-been-obtained}

\begin{itemize}
\tightlist
\item
  The Olympic data was downloaded from Kaggle.
\item
  GDP and population datasets were downloaded from the World Bank's data
  portal.
\end{itemize}

\subsubsection{Description of the
Question}\label{description-of-the-question}

We aim to study: - Whether wealthier countries (by GDP per capita) win
more Olympic medals. - How population size affects medal counts per
capita.

\subsection{2. Methodology}\label{methodology}

\subsubsection{Data Clean-up Procedures}\label{data-clean-up-procedures}

\begin{itemize}
\tightlist
\item
  Dropped irrelevant columns from the Olympic dataset (\texttt{Season},
  \texttt{Games}).
\item
  Renamed columns for consistency.
\item
  Converted GDP and population data from wide to long format.
\item
  Filtered out invalid countries (e.g., mixed teams, regions).
\item
  Merged datasets using country name and year.
\end{itemize}

\subsubsection{Scientific Workflow}\label{scientific-workflow}

\begin{enumerate}
\def\labelenumi{\arabic{enumi}.}
\tightlist
\item
  Clean datasets.
\item
  Convert and align formats.
\item
  Merge on common keys.
\item
  Analyze medal counts vs GDP/population.
\item
  Visualize results.
\end{enumerate}

\subsubsection{Data Representation
Choices}\label{data-representation-choices}

\begin{itemize}
\tightlist
\item
  Used log-log plots to handle skewed data distributions.
\item
  Boxplots to show relative medals per million by population group.
\item
  Used color scales and labels to highlight outliers.
\end{itemize}

\subsection{3. Data Preparation}\label{data-preparation}

We first start by reading the input data

we will now get rid of redundant columns in the olympics dataset ( by
removing the columns season and games)

then we change the format of the two datasets taken from world bank data
to put them in long format instead of the original wide format

\subsection{4. Merging Datasets}\label{merging-datasets}

after filtering the countries using a list from the gdp and populations
datasets since the column team in athletes dataset sometimes has non
country elements , we will then merge the three datasets left joining
using the year and the country which corresponds to the column team in
the athlets dataset

after merging the data and in order to make the first graphs we ll have
to filter everytime what s is not available such as the GDP and
population

\subsection{5. Analysis}\label{analysis}

\subsubsection{5.1 GDP vs Total Medals}\label{gdp-vs-total-medals}

\includegraphics{projet_MSPL_files/figure-latex/unnamed-chunk-5-1.pdf}

the relationship between GDP per capita and the total number of Olympic
medals won by countries. Overall, there is a weak but noticeable
positive trend, suggesting that countries with higher GDP per capita
tend to win more medals. However, the correlation is not strong
countries with modest GDP still achieve high medal counts, likely due to
strong sports infrastructure or cultural emphasis on athletics.
Conversely, many wealthy countries cluster with low to moderate medal
counts. This implies that while economic wealth can support athletic
success, it is not the only determining factor

\subsubsection{5.2 Medals per Million by Population
Group}\label{medals-per-million-by-population-group}

\includegraphics{projet_MSPL_files/figure-latex/unnamed-chunk-6-1.pdf}
From the graph, we can observe that that smaller countries tend to win
more medals per million people than larger ones. While large populations
offer a bigger talent pool, they don't guarantee Olympic success.
Medium-sized and small countries often perform better relative to their
population, likely due to stronger sports programs or focused
investment.

countries with mid-sized populations usualy are the developed countries
such as europe japan and north america that explains the important
number of medals The only major outlier beyond the 1 billion population
mark in Olympic medal count is likely China, as it consistently secures
a high number of medals, unlike India which, despite a similarly large
population, contributes fewer medals overall.

\subsubsection{5.3 Highlighting Top
Countries}\label{highlighting-top-countries}

\includegraphics{projet_MSPL_files/figure-latex/unnamed-chunk-7-1.pdf}

we can conclude that countries with higher GDP per capita generally win
more Olympic medals, especially those with moderate to large
populations. While most high-population countries cluster at lower medal
counts, a few exceptions---likely including China and the U.S.---stand
out with both high population and high medal tallies. Overall, GDP per
capita appears to be a stronger a stronger indicator of Olympic success
than population alone.

\subsubsection{5.4 Analyzing cultural differences and cost of
practice}\label{analyzing-cultural-differences-and-cost-of-practice}

The total costs and investments needed to practice each sport vary
wildly, as the figures below illustrates, some sports like Skiing and
Tennis require more equipment and higher registration costs, other like
athletics are alot more accessible to all socio economic categories of a
population, this has a direct result on the performance in the Olympics
for each sport as the number of sports accessible and availible to the
populations of first world countries is alot bigger than the number of
sports available to developing and third world countries.

On the basis of the infograpghic we selected a number of ``coslty''
sports and ``cheap'' sports, calculated the median GDP and visualize the
performance of each category for these selected sports.

We'll use a bar plot which allows direct side-by-side comparison within
each sport category, making disparities immediately visible, the log
scale accommodates the wide range of medal counts and makes relative
differences more apparent, finally, dividing by the number of countries
in each category creates a fair comparison, removing the bias from
having more developing countries than developed ones

\includegraphics{projet_MSPL_files/figure-latex/unnamed-chunk-9-1.pdf}

We can see that while first world countries perform better than
developing countries across the board as expected, developing countries
struggle alot more in ``costly'' sports such as Tennis and Table Tennis
than in Athletics and Judo,

We note also that the number of medals per country for Table Tennis is
barely higher than one,due primarily to the dominance of a handful of
east asian countries (mainly China), which suggests that the cultural
aspect plays a much bigger role in some sports regardless of the
investments needded.

\subsection{6. Conclusion}\label{conclusion}

There is a weak positive relationship between GDP per capita and Olympic
medal counts, suggesting that economic strength helps but is not the
sole predictor of success. Smaller and mid-sized countries often perform
better per capita due to more focused programs.

This analysis highlights that while some measurable and quantifiable
metrics such as wealth and population have an influence on the
performances of a country in the Olympics, they alone do not guarantee
dominance, other non quantifiable factors (cultural, political, and
structural\ldots etc) also play key roles.

Also, factors related to the Olympics themselves that aren't necessarily
related to the participating countries (rules, admission of
professionals, sports selected).

\subsection{7. References}\label{references}

\begin{itemize}
\tightlist
\item
  Olympic Data:
  \url{https://www.kaggle.com/datasets/heesoo37/120-years-of-olympic-history-athletes-and-results}
\item
  World Bank GDP/Population: \url{https://data.worldbank.org}
\end{itemize}

\end{document}
